\documentclass[11pt]{article}
\usepackage{natbib}
\usepackage{setspace}
\usepackage{caption}
\usepackage{subcaption}
\usepackage{graphicx}
\usepackage{multirow}
\usepackage[margin=0.9in]{geometry}
\usepackage[dvipsnames]{xcolor}
\usepackage[utf8]{inputenc}
\usepackage{tikz}
\usepackage{courier}
\usepackage{booktabs}
\usepackage{placeins}
\usepackage{longtable}
\usepackage{tabularx}
\usepackage{makecell}
\usepackage{indentfirst}
\usepackage{xcolor}
\usepackage{rotfloat}
\usepackage{booktabs}
\usepackage{threeparttable}
\usepackage{adjustbox}
\usepackage{bbm}
\usepackage{amsmath}
\usepackage{comment}
\usepackage{todonotes}


\title{Problem Set 1}
\author{Ana, Daniela, Rafael}
\date{\today}

\begin{document}

\maketitle

\section*{Productivity Estimation}

\subsection*{Question 1}

\begin{table}[htbp]\centering
\def\sym#1{\ifmmode^{#1}\else\(^{#1}\)\fi}
\caption{Summary Statistics for the Full Sample \label{tab:fullstats}}
\begin{tabular}{l*{1}{cccccccc}}
\toprule
                    &        Mean&          SD&         Min&    Perc. 25&      Median&    Perc. 75&         Max&           N\\
\midrule
Log of Output       &       13.49&         1.7&        5.91&       12.42&       13.59&       14.66&       19.16&      39,569\\
Log of Labor        &        5.00&         1.0&        0.62&        4.33&        5.01&        5.68&        8.86&      39,569\\
Log of Investment   &        5.03&         1.0&        1.13&        4.37&        5.03&        5.71&        9.34&      39,569\\
Log of Capital      &        8.99&         1.9&        2.09&        7.99&        9.29&       10.29&       14.57&      39,569\\
Age of the firm     &        8.54&         3.2&        1.00&        6.00&        9.00&       11.00&       17.00&      39,569\\
\bottomrule
\end{tabular}
\end{table}

\begin{table}[htbp]\centering
\def\sym#1{\ifmmode^{#1}\else\(^{#1}\)\fi}
\caption{Summary Statistics for the Balanced Sample \label{tab:balstats}}
\begin{tabular}{l*{1}{cccccccc}}
\toprule
                    &        Mean&          SD&         Min&    Perc. 25&      Median&    Perc. 75&         Max&           N\\
\midrule
Log of Output       &       13.41&         1.7&        5.91&       12.36&       13.52&       14.57&       18.87&      21,800\\
Log of Labor        &        4.99&         1.0&        1.10&        4.32&        5.00&        5.67&        8.86&      21,800\\
Log of Investment   &        5.04&         1.0&        1.13&        4.37&        5.04&        5.73&        9.34&      21,800\\
Log of Capital      &        9.16&         1.8&        2.24&        8.26&        9.43&       10.39&       14.34&      21,800\\
Age of the firm     &        7.32&         3.2&        1.00&        5.00&        7.00&       10.00&       16.00&      21,800\\
\bottomrule
\end{tabular}
\end{table}

\begin{table}[htbp]\centering
\def\sym#1{\ifmmode^{#1}\else\(^{#1}\)\fi}
\caption{Summary Statistics for the Exiters Sample \label{tab:exitstats}}
\begin{tabular}{l*{1}{cccccccc}}
\toprule
                    &        Mean&          SD&         Min&    Perc. 25&      Median&    Perc. 75&         Max&           N\\
\midrule
Log of Output       &       13.59&         1.7&        6.71&       12.51&       13.69&       14.77&       19.16&      17,769\\
Log of Labor        &        5.01&         1.0&        0.62&        4.34&        5.01&        5.69&        8.60&      17,769\\
Log of Investment   &        5.02&         1.0&        1.34&        4.37&        5.02&        5.70&        8.87&      17,769\\
Log of Capital      &        8.78&         1.9&        2.09&        7.66&        9.11&       10.15&       14.57&      17,769\\
Age of the firm     &       10.03&         2.5&        1.00&        8.00&       10.00&       12.00&       17.00&      17,769\\
\bottomrule
\end{tabular}
\end{table}

\FloatBarrier

\newpage
\subsection*{Question 2}
\begin{table}[htbp]\centering
\def\sym#1{\ifmmode^{#1}\else\(^{#1}\)\fi}
\caption{Total, Between, Within and Random Effects Estimators \label{tab:q2}}
\begin{tabular}{l*{4}{c}}
\toprule
                    &\multicolumn{1}{c}{(1)}&\multicolumn{1}{c}{(2)}&\multicolumn{1}{c}{(3)}&\multicolumn{1}{c}{(4)}\\
                    &\multicolumn{1}{c}{Pooled}&\multicolumn{1}{c}{Between}&\multicolumn{1}{c}{Within}&\multicolumn{1}{c}{Random Effects}\\
\midrule
Age of the firm     &       0.133\sym{***}&       0.128\sym{***}&       0.188\sym{***}&       0.133\sym{***}\\
                    &     (0.005)         &     (0.006)         &     (0.006)         &     (0.006)         \\
\addlinespace
Log of Capital      &       0.431\sym{***}&       0.555\sym{***}&       0.388\sym{***}&       0.421\sym{***}\\
                    &     (0.007)         &     (0.016)         &     (0.008)         &     (0.007)         \\
\addlinespace
Log of Labor        &       0.594\sym{***}&       0.613\sym{***}&       0.592\sym{***}&       0.594\sym{***}\\
                    &     (0.008)         &     (0.030)         &     (0.008)         &     (0.008)         \\
\midrule
Observations        &       21800         &       21800         &       21800         &       21800         \\
\bottomrule
\end{tabular}
\end{table}

\FloatBarrier

\subsection*{Question 3}
\begin{table}[htbp]\centering
\def\sym#1{\ifmmode^{#1}\else\(^{#1}\)\fi}
\caption{Difference Estimators \label{tab:q3}}
\begin{tabular}{l*{3}{c}}
\toprule
                    &\multicolumn{1}{c}{(1)}&\multicolumn{1}{c}{(2)}&\multicolumn{1}{c}{(3)}\\
                    &\multicolumn{1}{c}{First}&\multicolumn{1}{c}{Second}&\multicolumn{1}{c}{Third}\\
\midrule
Age of the firm     &       0.139\sym{***}&       0.101\sym{***}&       0.183\sym{***}\\
                    &     (0.013)         &     (0.009)         &     (0.008)         \\
\addlinespace
Log of Capital      &       0.252\sym{***}&       0.372\sym{***}&       0.399\sym{***}\\
                    &     (0.011)         &     (0.009)         &     (0.009)         \\
\addlinespace
Log of Labor        &       0.593\sym{***}&       0.595\sym{***}&       0.573\sym{***}\\
                    &     (0.009)         &     (0.009)         &     (0.009)         \\
\midrule
Observations        &       15260         &       15260         &       15260         \\
\bottomrule
\end{tabular}
\end{table}

\FloatBarrier

\newpage
\subsection*{Question 4}
\subsubsection*{(a)}
\begin{table}[htbp]\centering
\def\sym#1{\ifmmode^{#1}\else\(^{#1}\)\fi}
\caption{Total and Within Estimators for Full Sample \label{tab:q4a}}
\begin{tabular}{l*{2}{c}}
\toprule
                    &\multicolumn{1}{c}{(1)}&\multicolumn{1}{c}{(2)}\\
                    &\multicolumn{1}{c}{Pooled}&\multicolumn{1}{c}{Within}\\
\midrule
Age of the firm     &       0.133\sym{***}&       0.198\sym{***}\\
                    &     (0.002)         &     (0.005)         \\
\addlinespace
Log of Capital      &       0.414\sym{***}&       0.362\sym{***}\\
                    &     (0.005)         &     (0.006)         \\
\addlinespace
Log of Labor        &       0.597\sym{***}&       0.594\sym{***}\\
                    &     (0.006)         &     (0.006)         \\
\midrule
N                   &      39,569         &      39,569         \\
\bottomrule
\end{tabular}
\end{table}

\FloatBarrier

\subsubsection*{(b)}
\begin{table}[htbp]\centering
\def\sym#1{\ifmmode^{#1}\else\(^{#1}\)\fi}
\caption{Probit Model for Exiting Probability}
\begin{tabular}{l*{1}{c}}
\toprule
                    &\multicolumn{1}{c}{(1)}\\
                    &\multicolumn{1}{c}{Continuation Dummy}\\
\midrule
Log of Investment   &       0.477\sym{***}\\
                    &     (0.015)         \\
\addlinespace
Age of the firm     &      -0.692\sym{***}\\
                    &     (0.012)         \\
\addlinespace
Log of Capital      &       0.196\sym{***}\\
                    &     (0.011)         \\
\midrule
Observations        &       39569         \\
\bottomrule
\end{tabular}
\end{table}

\begin{table}[htbp]\centering
\def\sym#1{\ifmmode^{#1}\else\(^{#1}\)\fi}
\caption{Total and Within Estimators correcting for Selection}
\begin{tabular}{l*{2}{c}}
\toprule
                    &\multicolumn{1}{c}{(1)}&\multicolumn{1}{c}{(2)}\\
                    &\multicolumn{1}{c}{Pooled}&\multicolumn{1}{c}{Within}\\
\midrule
Age of the firm     &       0.130\sym{***}&       0.199\sym{***}\\
                    &     (0.003)         &     (0.005)         \\
\addlinespace
Log of Capital      &       0.415\sym{***}&       0.363\sym{***}\\
                    &     (0.005)         &     (0.006)         \\
\addlinespace
Log of Labor        &       0.597\sym{***}&       0.594\sym{***}\\
                    &     (0.006)         &     (0.006)         \\
\midrule
Observations        &       39569         &       39569         \\
\bottomrule
\end{tabular}
\end{table}

\FloatBarrier

\subsection*{Question 5}
\subsubsection*{(a)}
\begin{table}[htbp]\centering
\def\sym#1{\ifmmode^{#1}\else\(^{#1}\)\fi}
\caption{OP First Stage}
\begin{tabular}{l*{1}{c}}
\toprule
                    &\multicolumn{1}{c}{(1)}         \\
\midrule
Log of Labor        &       0.598\sym{***}\\
                    &     (0.006)         \\
\midrule
Observations        &       39569         \\
\bottomrule
\end{tabular}
\end{table}

\FloatBarrier
The coefficient obtained for labor here is similar to the one obtained in the question above, suggesting that endogeneity might not be such a big concern. 

\subsubsection*{(c)}
\begin{table}[htbp]\centering
\def\sym#1{\ifmmode^{#1}\else\(^{#1}\)\fi}
\caption{OP Second Stage}
\begin{tabular}{l*{1}{c}}
\toprule
                    &\multicolumn{1}{c}{(1)}         \\
\midrule
Log of Capital      &       0.271\sym{***}\\
                    &     (0.010)         \\
\addlinespace
Age of the firm     &       0.147\sym{***}\\
                    &     (0.006)         \\
\midrule
Observations        &       34672         \\
\bottomrule
\end{tabular}
\end{table}

\FloatBarrier
Comparing our results to the ones found in questions 2 and 4, we can see that the coefficient for capital (0.271) is lower than the one found in the all four specifications analyzed in question 2. The same applies to question 4. As expected, all coefficients are positive -- firms that suffer higher productivity shocks hire more capital. 
With respect to the coefficient for age of the firm (0.147), we verify the opposite. All coefficients found in questions 2 and 4 are lower for age than the one we found in the second-stage, but the differences are not as stark as for the capital case. 
This seems to suggest that endogeneity is a concern for the coefficient of capital. In addition, this model should also correct for measurement error.  
\newpage
\subsubsection*{(d)}
\begin{table}[htbp]\centering
\def\sym#1{\ifmmode^{#1}\else\(^{#1}\)\fi}
\caption{OP Second Stage correcting for Selection \label{tab:q5d}}
\begin{tabular}{l*{1}{c}}
\toprule
                    &\multicolumn{1}{c}{(1)}         \\
\midrule
Log of Capital      &       0.303\sym{***}\\
                    &     (0.013)         \\
\addlinespace
Age of the firm     &       0.113\sym{***}\\
                    &     (0.006)         \\
\midrule
N                   &      34,672         \\
\bottomrule
\end{tabular}
\end{table}

\FloatBarrier
After correcting for selection, our coefficient for capital rises, while the one for age decreases. 
This result makes sense, as selection should lead to a negative bias in the capital coefficient. 
\subsubsection*{(e)}
\begin{table}[htbp]\centering
\def\sym#1{\ifmmode^{#1}\else\(^{#1}\)\fi}
\caption{OP Estimation correcting for Endogeneity and Selection \label{tab:q5e}}
\begin{tabular}{l*{1}{c}}
\toprule
                    &\multicolumn{1}{c}{(1)}\\
                    &\multicolumn{1}{c}{} \\
\midrule
Age of the firm     &       0.117\sym{***}\\
                    &     (0.006)         \\
\addlinespace
Log of Capital      &       0.292\sym{***}\\
                    &     (0.010)         \\
\addlinespace
Log of Labor        &       0.598\sym{***}\\
                    &     (0.005)         \\
\midrule
N                   &      39,569         \\
\bottomrule
\end{tabular}
\end{table}

\FloatBarrier
The results are very similar to what was found before. The steps we went through before were simply a decomposition of this command.  


\end{document}